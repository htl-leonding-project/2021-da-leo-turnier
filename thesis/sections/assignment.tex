In diesem Kapitel wird beschrieben, um was es in der Diplomarbeit geht und was damit erreicht werden soll.

\section{Ausgangssituation}
Die HTL Leonding ist eine Höhere Technische Lehranstalt. 
Derzeit wird sie von ca. 1000 Schülern besucht, welche in eine der 4 Zweige:
\begin{itemize}
\item Informatik
\item Medientechnik
\item Elektronik
\item Medizintechnik
\end{itemize}

unterrichtet werden. Die Schulen und Vereine in der Umgebung veranstalten 
verschiedenste Sportturniere in den unterschiedlichsten Sportarten und Turniermodi.

\section{Problemstellung}

 Derzeit werden an unserer Schule Sportturniere lediglich auf Papier festgehalten, 
 was es relativ umständlich zu verwalten macht und sehr viel Zeit in Anspruch nimmt. 
 Auch sich über den aktuellen Stand eines Tunrniers zu informieren ist nur mündlich oder an einer Pinnwand möglich.

\section{Aufgabenstellung}
Mit einer neuen Applikation wollen wir dieses bisherige Verfahren ersetzen und die Gestaltung 
und Verwaltung von Sportturnieren unserer Schule vereinfachen. Dabei soll die Applikation so gestaltet werden, 
dass sie mehrere Turnier-Modi unterstützt und auch unabhänigig von der Sportart ist.

\section{Ziele}
Im Rahmen dieser Diplomarbeit soll nun eine Applikation erstellt werden die das Verwalten und die 
Informationsbeschafung eines Turniers komplett digitalisiert und um ein Vielfaches beschleunigt und vereinfacht. 
Die Informationsbeschaffung soll für jede Person mit einem Internetzugang möglich sein, 
wobei das Verwalten nur ausgewählten Personen vorbehalten ist.

Nun da die Ziele der Diplomarbeit definiert sind, geht es weiter wit der Systemarchitektur.