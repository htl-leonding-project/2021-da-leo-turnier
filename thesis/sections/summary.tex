Während der Schulzeit arbeitet man zwar an vielen Projekten,
diese erforden jedoch im Vergleich zur Diplomarbeit weit nicht so viel Organisationsaufwand und Kommunikation, da die Diplomanden
für längere Zeit an dem selben Projekt arbeiten. Dabei war es vor allem wichtig, gerade auf Github immer die aktuellste Version zu haben und Erweiterungen mit
informativen Commit-Messages zu beschreiben. 

Von den verwendeten Technologie wurde schon mit Einigen während der Schulzeit gearbeitet, wie zum Beispiel Angular und Quarkus, in diese wurde sich im Zuge der Diplomarbeit noch deutlich vertieft.
Durch die getrennte Arbeitsaufteilung von Backend und Frontend, war die Technologien Postman und MockLab sehr wichtig um keine Unterbrechungen in der Entwickelung zu verursachen.
Als besonders große Herausforderung stellte sich die Arbeit mit Keycloak heraus, beziehungsweise die generelle Arbeit mit Authentifizierung und Autorisierung in einer größeren Applikation.

Die Diplomanden haben im Verlauf der Diplomarbeit viel Erfahrung im Bereich Turniere gemacht, wie sie aufgebaut sind, wie sie so spannend wie möglich gestaltet werden, 
usw.

Eine besonders schwere Aufgabe war es, anfangs die Ausmaße unseres Projektes einzuschätzen und im Voraus dafür zu planen. Dabei wurden vor allem das Datenmodell im Projektverlauf mehrmals überarbeitet.
Darum war es wichtig das der Code so geschrieben wurde, dass er auch nach längerer Zeit noch verständlich ist.

Mögliche Erweiterungen des Projekts könnten sein:

\begin{itemize}
    \item Live-Ticker der mehr Einsicht in das Spielgeschehen gibt
    \item Detailreicher Statisken über Spiele oder einzelne Teilnehmer
    \item Manuelle Anordnung der Seeds
    \item Ein Chat damit sich Zuschauer über das Spielgeschehen unterhalten können
\end{itemize}