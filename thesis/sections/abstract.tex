\begin{spacing}{1}
    \chapter*{Abstract}
\end{spacing}
A prototype application was developed to simplify the creation and execution of tournaments for the tournament organizer. 
The tournaments can be run regardless of the sport, and  
3 different tournament modes are enabled, KO, Round Robin and Combination.
KO means the loser is eliminated from the tournament and the winner goes to the next round. Round Robin means everyone 
plays against everyone, regardless of results. And Combination means the players are split into a cercain number of groups, which is 
defined by the admin or tournament organizer starting the tournament. In these Groups everyone plays everyone else of that same group, just like 
in Round Robin, and then a certain number of competitors with the most wins from each group, also defined by the admin or tournament 
organizer starting the tournament, move up to the KO phase, where the loser is eliminated and the winner plays the next round, just like in KO.

One can access the application either as an admin, tournament organizer or spectator.
As an admin you can change, add or delete any kind of data, the tournament organizer is 
responsible for everything concerning the execution of the tournament, and the spectator 
cannot change anything in the tournament, but only watch it.
\newpage
\begin{spacing}{1}
    \chapter*{Zusammenfassung}
\end{spacing}
Es wurde ein Prototyp einer Application entwickelt, die das Erstellen und Durchführen 
von Turnieren für den Turnierorganisator vereinfachen und rationalisieren soll.
Die Turniere können unabhängig von der Sportart durchgeführt werden, außerdem 
werden 3 verschiedene Turniermodi ermöglicht, KO, Round Robin und Combination.

KO bedeutet, dass der Verlierer aus dem Turnier ausscheidet und der Gewinner in die nächste Runde kommt. Round Robin bedeutet, dass jeder 
jeder gegen jeden spielt, unabhängig vom Ergebnis. Und Combination bedeutet, dass die Spieler in eine bestimmte Anzahl von Gruppen aufgeteilt werden, 
die der Admin oder der Turnierorganisator zu Beginn des Turniers festlegt. In diesen Gruppen spielt jeder gegen jeden aus der gleichen Gruppe, genau 
wie bei Round Robin, und dann steigt eine bestimmte Anzahl von Teilnehmern mit den meisten Siegen aus jeder Gruppe, die ebenfalls vom Admin oder Turnierorganisator festgelegt wird, 
in die KO-Phase auf, in der der Verlierer ausscheidet und der Gewinner in die nächste Runde kommt, genau wie im KO-System.

Man Kann auf die Applikation entweder als Admin, Turnierorganisator oder Zuschauer zugreifen.
Als Admin kann man jegliche Art von Daten verändern, hinzufügen oder löschen, der Turnierorganisator ist 
für alles, was die Durchführung der Turniere angeht, verantwortlich, und der Zuschauer 
kann am Turnier nichts verändern, sondern nur zuschauen.
\lipsum[0]
