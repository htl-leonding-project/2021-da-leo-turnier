\section{Verwirklichung der Anforderungen}

\begin{figure}[H]
    \centering
    \caption{system architecture}
    \includegraphics[scale=0.25]{pics/system_architecture.png}
\end{figure}


\section{Verwendete Technologien}

\subsection{Git}
\includegraphics[scale=0.05]{pics/git.png}

Git ist das mit Abstand am weitesten verbreitete Versionskontrollsystem der Welt. Der Name Git wird aus der britischen 
Umgangssprache übersetzt und bedeutet "Blödmann.
\cite{sysarch-git-1}
Es ist ein Open-Source Projekt, 
das ursprünglich 2005 von dem Entwickler des Linux Betriebsystem-Kernels entwickelt wurde. Es ist außerdem mit sowohl mit 
Windows als auch Linux Systemen kompatibel und auch in vielen verschiedenen IDEs integriert. Git ist ein verteiltes Versionskontrollsystem, 
was bedeutet, dass der Versionsverlauf nicht nur an einem Ort gespeichert ist, wie es bei älteren Versionskontrollsystemen der Fall war, 
sondern in jeder Arbeitskopie der gesamte Verlauf aller Änderungen im Repository (Aufbewahrungsort) enthalten sind. 

\cite{sysarch-git-2}

\subsubsection{Was ist ein Versionskontrollsystem?}

Ein Versionskontrollsystem wie Git wird in der Softwareentwicklung verwendet, 
um Änderungen des Quellcodes zu speichern und zu verwalten. 
Hier gibt es drei Arten von Versionskontrollsystemen:
\cite{sysarch-git-2}
\begin{itemize}
 \item lokale Versionsverwaltung (Local Version Control System - LVCS): 
\end{itemize}
Hier werden Dateien lokal einfach nur in ein separates Verzeichnis kopiert.
\cite{sysarch-git-2}
\begin{itemize}
 \item zentrale Versionsverwaltung (Centralized Version Control System - CVCS):
\end{itemize}
Hier gibt es einen einen zentralen Server, der alle versionierten Dateien verwaltet und es Personen ermöglicht, 
Dateien von diesem zentralen Repository abzuholen und auf ihren PC zu übertragen.
\cite{sysarch-git-2}
\begin{itemize}
 \item verteilten Versionsverwaltung (Distributed Version Control System - DVCS):
\end{itemize}
Wie weiter oben schon angesprochen ist diese Variante jene, die von Git benutzt wird. Hier gibt es zwar ein zentrales Repository, 
aber jede Person  hat eine Kopie des Repositories und somit die vollständigen Projektdaten.
\cite{sysarch-git-2}

\subsubsection{Git Befehle}

Git verwendet zum verwalten eines Repositories das Terminal, hierzu gibt es einige wichtige Befehle. Zuerst kann man mit "git init" ein leeres Repository erstellen 
oder ein existierendes nochmal initialisieren, alternativ kann man mit "git clone" ein existierendes Repository kopieren. Damit ein File von Git gespeichert werden
kann, muss man es zunächst mit "git add" zum Repository hinzufügen, das ganze kann dann mit "git rm" wieder rückgängig gemacht werden. Mit "git status" wird der 
momentane Status aller Files im Repository angezeigt, dass heißt, ob das File neu im Repository ist, ob es seit dem letzten Mal Speichern verändert wurde, oder ob es 
erst garnicht im Repository vorhanden ist. Nun kann man mit "git commit" den momentanen Status des Repositories als neue Version speichern und mit "git push" an ein "remote" 
Repository senden. Als letzter wichtiger Befehl gilt noch "git branch", hiermit kann man das Repository in verschiedene "Branches" aufteilen und so neue Features getrennt voneinander
entwickeln, und diese dann mit "git merge" im Hauptbranch zusammenfügen.
\cite{sysarch-git-2}

\subsection{GitHub}
\includegraphics[scale=0.05]{pics/github.png}

Github ist ein Cloud-basiertes Repository Hosting Service, das die verteilte Versionskontrolle von Git zur verfügung stellt. Es wurde 2008 von Chris Wanstrath, P. J. Hyett, 
Scott Chacon und Tom Preston-Werner gestartet. Zu dem Zeitpunkt war Git noch relativ unbekannt, wesshalb es noch keine anderen Optionen gab. Die Software wurde in der 
Programmiersprache "Ruby on Rails and Erlang" entwickelt. 
\cite{sysarch-github-1}
Das Ziel von Github ist es, eine benutzerfreundliche Oberfläche für Git zur verfügung zu stellen, mit der man auch mit weniger technischem Wissen die Vorteile von Git ausnutzen kann.
\cite{sysarch-github-2}
Als Unternehmen verdient GitHub Geld, indem es gehostete private Code-Repositories sowie andere geschäftsorientierte Pläne verkauft, 
die es Unternehmen erleichtern, Teammitglieder und Sicherheit zu verwalten.
\cite{sysarch-github-2}

\subsubsection{Github Issies}




\subsection{Intellij IDEA}
\includegraphics[scale=0.05]{pics/intellijIdeaLogo.png}


Zur Entwicklung des Java Backends verwenden wir die IDE Intellij IDEA in der Ultimate Edition vom tschechische 
Software Unternehmen Jetbrains. Intellij enthält eine Vielzahl an Tools welche das Arbeiten um einiges erleichtern, 
zum Beispiel einen Direkten Zugang zu Datenbanken aller Art wie die von uns verwendete PostgresSQL Datenbank 
oder die kompabilität mit allerlei Versionsverwaltungssytemen wie Github oder Bitbucket.

\subsection{WebStorm}
\includegraphics[scale=0.025]{pics/WebStormLogo.svg.png}


Obwohl es möglich wäre mit Intellij ein Angular Projekt zu entwickeln haben wir uns bei der Frontend Entwicklung 
für die IDE Webstorm entschieden. Diese ist genauso wie Intellij von Jetbrains doch enthält mehr support für die 
Programmiersprachen JavaScript und TypeScript, sowie einen Built-in Debugger für Client-Side JavaScript und Node.js 

\subsection{Quarkus}
\includegraphics[scale=0.015]{pics/quarkusLogo.png}

Quarkus ist ein Framework zur Erstellung von Java-Anwendungen mit dem Java speziell für Container optimiert wird. 
Es bietet eine effektive Plattform für Serverless-, Cloud- und Kubernetes Umgebungen. Der Hersteller Redhat wirbt 
auch mit schnellen Startzeiten und geringen Speicherplatzverbrauch. Dise erzielt Quarkus dadurch dass es den Code schon Buildvorgangs verabeitet.

\subsection{Maven}
\includegraphics[scale=0.015]{pics/apacheMavenLogo.svg.png}

Maven ist ein Build Atomation Tool welches hauptsächlich für Java verwendet wird und folgt dem Ansatz Konvention vor Konfiguration. TODO


\subsection{Angular}
\includegraphics[scale=0.02]{pics/angularLogo.svg.png}

Angular ist ein Client-Side JavaScript Framework zur Erstellung von Single-Page-Webapplications. 
Seine komponentenbasierte Architektur welche View und Logik trennt macht es für den Entwickler leicht 
Applikationen zu warten und testen. Dabei verwendet Angular TypeScript für den Code-Behind und HTML als Auszeichnungssprache. 
Die vielen gut integrierten Libraries decken Features wie Routing, Verwaltung von Formulare und Client-Server Kommunikation ab. 

\subsection{PostgreSQL}
\includegraphics[scale=0.015]{pics/postgresqlLogo.svg.png}

PostgreSQL ist eine objektrelationale Datenbank welche sich durch ihre Stabilität und freie Verfügbarkeit auszeichnet. 
Sie hält sich dabei sehr eng an den SQL-Standard. Es werden eine Vielzahl von Datentypen und Operatorn unterstützt und 
Entwickler können auch eigene Datentypen definieren. Ein weitetere großer Vorteil von PostgreSQL ist auch dass es auf 
jeder Hardware und auf beinahe jedem Betriebsystem läuft. 
\subsection{Docker}
\includegraphics[scale=0.05]{pics/dockerLogo.png}

Docker ist eine Software zur Containervirtualisierung . Die Container sind voneinander isoliert und haben ihre eigene Software, 
Libraries sowie Konfigurationsdatein. Kommunizieren können sie über vorher genau definierte Kanäle. Da alle Container 
auf dem selben OS-Kernel laufen brauchen sie weniger Resourcen wie eine Virtuele Maschine.